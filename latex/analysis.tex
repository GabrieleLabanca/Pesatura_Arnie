\section{Data analysis}
As it has been written in \autoref{topic:scale_tare}, the not-calibrated weight is elaborated on Thingspeak.com in order to obtain the true weight:
 
\label{code:ts:tare}
\begin{lstlisting}[language=Matlab]
%%%%%%%%%%%%%%%%%%%%%%%
% This code visualizes the true weight starting from raw readings (ch. 4),
%  starting from two readings to calculate the tare
% Averaged values!

% read/write variables 
readChId = ...;    // put your own keys here (see Thingspeak.com)
readKey  = '...';

% retrieve last 100 data from Fields 1 and 2
npoints = 100;
[dataWeight,timestamps] = thingSpeakRead(readChId,'NumPoints',npoints,'Fields',4);

% tare: the relation is assumed to be linear
tw0 = 0.0;      // the true weight put on the scale at time 0
rw0 = -318718;  // the value read on the scale when not-calibrated at time 0 
tw1 = 1.061;    // the true weight put on the scale at time 1
rw1 = -299786;  // the value read on the scale when not-calibrated at time 0 
k_tare = (tw1-tw0)/(rw1-rw0);
q_tare = tw1 - rw1*k_tare;

dataWeight = dataWeight*k_tare + q_tare;

% average
nmean = 40;
mean = 0.;
mean_dataWeight = [];
mean_timestamps = timestamps(1:npoints/nmean);
for i=1:size(dataWeight)
    mean = mean + dataWeight(i);
      if isequal(mod(i,nmean),0)
          mean_dataWeight(i/nmean) =  mean/nmean;
          mean_timestamps(i/nmean) = timestamps(i-nmean/2); % center of interval
          mean = 0.0;
      end       
end

%% Visualize Data %%
thingSpeakPlot(mean_timestamps,mean_dataWeight,'XLabel','tempo','YLabel','peso (kg)','Title','Peso arnia','Legend',{'peso'});
\end{lstlisting}

A very simple, yet satisfactory, approach has been taken to the elaboration of data: a sample over one or more days is considered, of which the weight-temperature points are fitted with a linear function. Assuming that the weight variation is neglectable with respect to the dependence on temperature of the response of the load cells, which is usually a decent assumption, the relation found corrects the instrumental noise:
\[ w_{\mathrm{vero}}[i] = w_{\mathrm{misurato}}[i] - (w_0 + m*T[i]) \]
It is worth noting that without correction the uncertainty on the average is of order $0.01 kg$, so depending on the aim of the project may be that no correction is necessary to obtain an acceptable result. 

The Matlab code, which can be directly used in a \textit{visualization} on Thingspeak.com, is presented (the weight is calibrated as in the previous code):
\begin{lstlisting}[language=Matlab]
% read/write variables
readChId = ...;
readKey  = '...';

% retrieve last 100 data from Fields 1 and 2
npoints = 500;%3*24*7;
[dataWeight,timestamps] = thingSpeakRead(readChId,'NumPoints',npoints,'Fields',4);%,'ReadKey',readKey);
dataTemperature = thingSpeakRead(readChId,'NumPoints',npoints,'Fields',2);%,'ReadKey',readKey);
dataHumidity = thingSpeakRead(readChId,'NumPoints',npoints,'Fields',3);

% transform raw data in true weights (see "tara")
tw0 = 0.0;
rw0 = -318718;
tw1 = 1.061;
rw1 = -299786;
k_tare = (tw1-tw0)/(rw1-rw0);
q_tare = tw1 - rw1*k_tare;

dataWeight = dataWeight*k_tare + q_tare;
oldW = dataWeight; % keeps old measure


% clean data, using correlation with Temperature
m = 0.01; % correlation coefficient
n_back = 2; % NB the temperature affects weight with a DELAY!
% T0:
%  set = 3 if want to compare old measures
%  set = dataTemperature(1) if only a estimate of the variation in last days is wanted
T0 = dataTemperature(1);
prevision = [];
for i=1:npoints
    if(i<(n_back+1))
            prevision(i) = dataWeight(1)-m*(dataTemperature(i)-T0);
    else
            prevision(i) = dataWeight(1)+m*(dataTemperature(i-n_back)-T0);
    end
    dataWeight(i) = dataWeight(i) - prevision(i);
end

% find mean values
nmean = 40;
mean = [0,0,0];
size(mean)
mean_dataWeight = [];
mean_oldW = [];
mean_prevision = [];
mean_timestamps = timestamps(1:npoints/nmean);
for i=1:npoints
    mean(1) = mean(1) + dataWeight(i);
    mean(2) = mean(2) + oldW(i);
    %mean(3) = mean(3) + prevision(i);
      if isequal(mod(i,nmean),0)
          mean_dataWeight(i/nmean) =  mean(1)/nmean;
          mean_oldW(i/nmean) = mean(2)/nmean;
          %mean_prevision(i/nmean) = mean(3)/nmean;
          mean_timestamps(i/nmean) = timestamps(i-nmean/2); % center of interval
          mean = [0,0];
      end       
end
numel(mean_dataWeight);
numel(mean_oldW);
numel(mean_timestamps);

%% Visualize Data %%
select = 'B'
if(select == 'A') % cfr old weight and temperature and prevision
    M = horzcat(dataTemperature,oldW,transpose(prevision));
    tax = timestamps; 
end
if(select == 'B') % THE INTERESTING GRAPH
    M = horzcat(transpose(mean_dataWeight),transpose(mean_oldW),mean_prevision);
    tax = mean_timestamps;
end
if(select == 'C') % cfr temperature and weight = > correlation
    M = oldW;
    tax = dataTemperature;
end 
thingSpeakPlot(tax,M,'XLabel','tempo','YLabel','peso (kg)','Title','Peso arnia (rispetto a peso originale)','Legend',{'corretto','misurato'});
\end{lstlisting}
